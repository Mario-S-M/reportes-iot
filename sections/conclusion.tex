\section{Conclusión}

La implementación del sensor ultrasónico HC-SR04 con la Raspberry Pi permite obtener mediciones de distancia de manera eficiente y precisa. A través del uso de los pines GPIO y la programación en Python, es posible realizar cálculos en tiempo real para medir distancias con una amplia variedad de aplicaciones, desde sistemas de detección de obstáculos hasta automatización industrial.

Durante el desarrollo del proyecto, se abordaron conceptos clave como el uso del divisor de voltaje para proteger los pines de la Raspberry Pi, la medición del tiempo de respuesta de la señal ultrasónica y su conversión en distancia. Además, se destacó la importancia del manejo de hardware mediante software embebido, lo que refuerza la integración entre la electrónica y la programación.

Este proyecto no solo permite comprender el funcionamiento de los sensores ultrasónicos, sino que también sirve como base para futuras aplicaciones en robótica, domótica y sistemas de monitoreo. Con una correcta implementación y optimización del código, es posible mejorar la precisión y la estabilidad de las mediciones, lo que hace de esta tecnología una herramienta versátil y útil en múltiples campos.

