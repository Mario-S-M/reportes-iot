\section{Conclusión}

En esta actividad logramos implementar un circuito básico de encendido y apagado de un LED mediante un botón utilizando una Raspberry Pi. A lo largo del desarrollo, comprendimos la importancia de las resistencias pull-down para evitar estados indeterminados en la entrada digital y reforzamos el uso de los pines GPIO para la interacción con hardware externo.

Este ejercicio no solo permitió reforzar conceptos fundamentales de electrónica y programación, sino que también sirvió como base para futuros proyectos en los que se requiera el control de dispositivos a través de una Raspberry Pi. Gracias a esta práctica, estamos mejor preparados para integrar componentes adicionales como sensores o actuadores, ampliando así las posibilidades de automatización y control en sistemas embebidos.
