\section{Conclusión}

Se ha logrado el control eficiente de un servomotor, LEDs y un buzzer mediante el uso de una Raspberry Pi. Este proyecto ha demostrado la capacidad de la Raspberry Pi para interactuar con múltiples dispositivos electrónicos y realizar tareas complejas de automatización.

El proceso de diseño, implementación y prueba de este sistema ha permitido adquirir un conocimiento profundo sobre la programación de la Raspberry Pi y su interacción con componentes electrónicos. Este tipo de proyectos resalta la importancia de la integración de hardware y software, mostrando cómo pueden colaborar para crear soluciones efectivas y funcionales.