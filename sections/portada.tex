% portada.tex
% Configuración del fondo solo para la portada
\backgroundsetup{
	scale=1,
	angle=0,
	opacity=0.2,
	contents={\includegraphics[width=\paperwidth,height=\paperheight]{imagenes/fondo_portada.png}}
}
\BgThispage

\begin{center}
	\vspace*{2cm}
	
	\Large\textbf{Actividad 1: Led y Boton de encendido y apagado}
	
	\vspace{1.5cm}
	
	Autor(es):
	
	\vspace{0.5cm}
	
	Luis Fernando Chávez Martínez
	
	Luis Javier Ramírez Arias
	
	Mario Eduardo Sánchez Mejía
	
	
	\vspace{1.5cm}
	
	Docente:
	
	\vspace{0.5cm}
	
	
	Juan Jesús Ruiz Lagunas
	
	\vspace{0.5cm}
	
	\begin{abstract}
		\noindent
		\justifying
		Esta actividad consiste en el control de un LED mediante un botón utilizando una Raspberry Pi. Se implementa un sistema de encendido y apagado que permite al usuario interactuar con el LED a través del botón, utilizando los pines GPIO de la Raspberry Pi. Se abordan conceptos básicos de electrónica, como el uso de resistencias pull-up y pull-down, y se desarrolla un script en Python para gestionar la lógica de funcionamiento. Esta práctica permite familiarizarse con la manipulación de hardware en sistemas embebidos y su aplicación en proyectos de automatización.
		
		\vspace{0.5cm}
		\noindent
		\textbf{Palabras clave:} Raspberry Pi, GPIO, LED, botón, encendido/apagado, Python, electrónica básica, sistemas embebidos, automatización.
	\end{abstract}
\end{center}

% Desactivar el fondo para las siguientes páginas
\clearpage
\backgroundsetup{contents={}}