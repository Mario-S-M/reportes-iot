% portada.tex
% Configuración del fondo solo para la portada
\backgroundsetup{
	scale=1,
	angle=0,
	opacity=0.2,
	contents={\includegraphics[width=\paperwidth,height=\paperheight]{imagenes/fondo_portada.png}}
}
\BgThispage

\begin{center}
	\vspace*{2cm}
	
	\Large\textbf{Actividad 2: Uso de servomotor, leds y buzzer}
	
	\vspace{1.5cm}
	
	Autor(es):
	
	\vspace{0.5cm}
	
	Luis Fernando Chávez Martínez
	
	Luis Javier Ramírez Arias
	
	Mario Eduardo Sánchez Mejía
	
	
	\vspace{1.5cm}
	
	Docente:
	
	\vspace{0.5cm}
	
	
	Juan Jesús Ruiz Lagunas
	
	\vspace{0.5cm}
	
	\begin{abstract}
		\noindent
		\justifying
		En esta práctica, se utiliza una Raspberry Pi para controlar un servomotor que simula una puerta. El objetivo es abrir y cerrar la puerta a ciertos ángulos, utilizando un par de LEDs de color distinto para conocer en qué posición se encuentra el servo. Además, el uso de un buzzer que emite un sonido al abrir la puerta. El sistema es desarrollado exitosamente, demostrando la capacidad de la Raspberry Pi para simular un entorno real y funcional.
		
		\vspace{0.5cm}
		\noindent
		\textbf{Palabras clave:} Raspberry Pi, GPIO, LED, servomotor, buzzer, Python, electrónica, automatización.
	\end{abstract}
\end{center}

% Desactivar el fondo para las siguientes páginas
\clearpage
\backgroundsetup{contents={}}