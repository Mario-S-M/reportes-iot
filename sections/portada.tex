% portada.tex
% Configuración del fondo solo para la portada
\backgroundsetup{
	scale=1,
	angle=0,
	opacity=0.2,
	contents={\includegraphics[width=\paperwidth,height=\paperheight]{imagenes/fondo_portada.png}}
}
\BgThispage

\begin{center}
	\vspace*{2cm}
	
	\Large\textbf{Actividad 3: Sensor Ultrasonico}
	
	\vspace{1.5cm}
	
	Autor(es):
	
	\vspace{0.5cm}
	
	Luis Fernando Chávez Martínez
	
	Luis Javier Ramírez Arias
	
	Mario Eduardo Sánchez Mejía
	
	
	\vspace{1.5cm}
	
	Docente:
	
	\vspace{0.5cm}
	
	
	Juan Jesús Ruiz Lagunas
	
	\vspace{0.5cm}

	\begin{abstract}
		\noindent
		\justifying
		Esta actividad consiste en la medición de distancia utilizando un sensor ultrasónico HC-SR04 con una Raspberry Pi. Se establece una comunicación entre el sensor y la Raspberry Pi a través de los pines GPIO, empleando un divisor de voltaje para adaptar la señal de retorno del sensor. La medición se basa en el tiempo de respuesta del pulso ultrasónico reflejado por un objeto, permitiendo calcular la distancia en centímetros. Se desarrolla un script en Python para gestionar la adquisición de datos y su procesamiento. Esta práctica permite comprender la integración de sensores en sistemas embebidos y su aplicación en proyectos de automatización y robótica.
		
		\vspace{0.5cm}
		\noindent
		\textbf{Palabras clave:} Raspberry Pi, HC-SR04, GPIO, sensor ultrasónico, medición de distancia, Python, sistemas embebidos, automatización, robótica.
	\end{abstract}
\end{center}

% Desactivar el fondo para las siguientes páginas
\clearpage
\backgroundsetup{contents={}}