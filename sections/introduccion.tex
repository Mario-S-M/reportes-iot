\section{Introducción}
El control de dispositivos electrónicos mediante microcomputadoras es una habilidad esencial en el desarrollo de proyectos de automatización y sistemas embebidos. La Raspberry Pi, gracias a sus pines GPIO (General Purpose Input/Output), permite la interacción con diversos componentes electrónicos, como sensores, motores y, en este caso, un LED y un botón.

En esta actividad, se implementará un sistema de encendido y apagado de un LED utilizando un botón como interruptor, lo que servirá para comprender el funcionamiento de los pines GPIO y la lógica de control digital. Se explorará la importancia de las resistencias pull-up y pull-down para evitar lecturas erróneas en el botón, además de la programación en Python para gestionar la entrada y salida de señales.

Este ejercicio representa un primer paso hacia el desarrollo de proyectos más complejos, como interfaces hombre-máquina, automatización de procesos y control remoto de dispositivos. A través de esta actividad, se reforzarán conocimientos en electrónica básica y programación, sentando las bases para futuros desarrollos en el mundo del hardware y la Internet de las Cosas (IoT).
\newpage