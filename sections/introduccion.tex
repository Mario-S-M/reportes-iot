\section{Introducción}
\noindent
\justifying
Los sensores ultrasónicos son ampliamente utilizados en aplicaciones de medición de distancia, detección de obstáculos y sistemas de navegación en robótica. El sensor HC-SR04 es un dispositivo de bajo costo que permite medir distancias mediante la emisión y recepción de ondas ultrasónicas, calculando el tiempo de retorno del eco para determinar la distancia a un objeto.

En esta práctica, se implementa un sistema de medición de distancia con un sensor HC-SR04 conectado a una Raspberry Pi. Se configura la comunicación a través de los pines GPIO y se emplea un divisor de voltaje para garantizar la compatibilidad entre la señal de retorno del sensor (5V) y los pines de la Raspberry Pi (3.3V). Se desarrolla un script en Python para enviar pulsos de activación al sensor, capturar la señal de respuesta y calcular la distancia en centímetros.

Este trabajo permite familiarizarse con la integración de sensores en sistemas embebidos, comprendiendo los principios básicos de la medición ultrasónica y su aplicación en proyectos de automatización, control de dispositivos y robótica. Además, refuerza el uso de la Raspberry Pi como una plataforma versátil para el desarrollo de soluciones tecnológicas.
