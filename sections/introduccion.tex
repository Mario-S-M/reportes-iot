\section{Introducción}

En esta práctica, se emplea una Raspberry Pi para controlar un servomotor que simula una puerta. Este proyecto se centra en la implementación de un sistema que abre y cierra la puerta en respuesta a comandos específicos. Además, se utilizan indicadores visuales y auditivos, como un LED verde para indicar la apertura y un LED rojo para la clausura, junto con un buzzer que emite un sonido al abrir la puerta.

El objeto de esta práctica es demostrar cómo la combinación de hardware y software puede ser utilizada para desarrollar soluciones innovadoras y prácticas. A través de la programación en Python, se logra un control preciso y confiable del servomotor y los dispositivos asociados, proporcionando una experiencia de aprendizaje integral y práctica para la materia de Internet de las Cosas.

\newpage