\section{Desarrollo}
Para llevar a cabo esta actividad, es fundamental comprender cómo funcionan los pines GPIO de la Raspberry Pi y su capacidad para interactuar con dispositivos electrónicos. Un LED es un componente que emite luz cuando recibe corriente, mientras que un botón actúa como un interruptor que permite o bloquea el flujo de corriente. Mediante la programación, se controlará el encendido y apagado del LED en respuesta a la pulsación del botón.

\subsection{Materiales necesarios}
Para esta práctica, se requieren los siguientes componentes:
\begin{itemize}
	\item \textbf{Raspberry Pi} (cualquier modelo con pines GPIO disponibles).
	\item \textbf{LED} (color opcional).
	\item \textbf{Resistencia de 330\textohm} (para limitar la corriente del LED y evitar daños).
	\item \textbf{Botón pulsador}.
	\item \textbf{Resistencia de 10k\textohm} (pull-down para evitar lecturas erróneas).
	\item \textbf{Cables de conexión}.
	\item \textbf{Protoboard} (para facilitar las conexiones).
\end{itemize}

\subsection{Conexión del circuito}
La conexión de los componentes se realizará de la siguiente manera:
\begin{itemize}
	\item El \textbf{ánodo (+) del LED} se conecta a un \textbf{pin GPIO de la Raspberry Pi} a través de la resistencia de 330\textohm.
	\item El \textbf{cátodo (-) del LED} se conecta a \textbf{GND} (tierra).
	\item Uno de los pines del \textbf{botón} se conecta a otro \textbf{pin GPIO} de la Raspberry Pi.
	\item El otro pin del botón se conecta a \textbf{GND}.
	\item Se coloca una \textbf{resistencia pull-down de 10k\textohm} entre el pin GPIO del botón y GND para asegurar una lectura estable.
\end{itemize}

\subsection{Explicación del funcionamiento}
\begin{enumerate}
	\item Se configura la Raspberry Pi para interpretar el estado del botón y controlar el LED en consecuencia.
	\item Cuando el botón se presiona, se cambia el estado del LED (encendido/apagado).
	\item Se utiliza un retraso en la programación para evitar lecturas incorrectas causadas por el rebote mecánico del botón.
	\item Al finalizar la ejecución, se restablecen los pines GPIO a su estado original para evitar conflictos en futuras conexiones.
\end{enumerate}

\subsection{Pruebas y resultados}
Para verificar el funcionamiento del sistema:
\begin{itemize}
	\item Conectar todos los componentes correctamente.
	\item Ejecutar el programa en la Raspberry Pi.
	\item Pulsar el botón y observar cómo el LED cambia de estado (se enciende o apaga con cada pulsación).
\end{itemize}

Si el LED no responde correctamente, se deben revisar las conexiones y asegurarse de que las resistencias estén en su lugar.

\subsection{Aplicaciones y mejoras}
Este sistema básico puede extenderse con diversas mejoras, como:
\begin{itemize}
	\item \textbf{Uso de interrupciones GPIO} en lugar de una lectura continua para optimizar el rendimiento.
	\item \textbf{Implementación de múltiples botones} para controlar diferentes acciones.
	\item \textbf{Conexión a una interfaz web o control remoto}, integrando tecnologías como Flask o MQTT para manejar el LED desde otro dispositivo.
	\item \textbf{Automatización avanzada}, como encender el LED solo en ciertos horarios o en respuesta a sensores externos.
\end{itemize}

\begin{figure}[h]
	\centering
	\includegraphics[width=0.5\textwidth]{C:/Users/mayit/Downloads/Boton Led 1}
	\caption{Botón LED 1}
\end{figure}

\begin{figure}[h]
	\centering
	\includegraphics[width=0.5\textwidth]{C:/Users/mayit/Downloads/Boton Led 2}
	\caption{Botón LED 2}
\end{figure}

\newpage